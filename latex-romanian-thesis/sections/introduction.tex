\chapter{Introduction}

\section{Purpose}

Given the increase use of artificial intelligence solutions in solving many of our current day-to-day problems, we decided to tackle the problem of face-based authentication in the systems that we use daily.

The challenge can be presented as follows: given a device that has access to a camera and which has been previously setup with a set of subject identities, the task is to validate the face in front of the camera as being known or not and in the positive case, the device should unlock itself.

The problems that arise are multiple and difficult to solve. Here are a few of them: firstly, the face in front of the camera should be validated as being real, therefore we need to give special attention to combating face spoofing attacks as printed photos, video replays or 3D masks. Secondly, the identification part of the system should be accurate enough so that two people with close facial features can be differentiated.

\section{Impact}
A successful implementation of such a solution could be deployed to border control system which would help amend the problem of crowded airports, could be used in process of card payments as a method of reliable authentication and also could be used as an important part of an Artificial Intelligence assistant in order to create a more human-like experience.

\section{State of the art}
------TODO: Gotta look into this----------

Conținut secțiune, exemplu citare \cite{hoare_csp}.
\textit{Cuvânt pentru} \textbf{glosar} \index{glosar}.	

\begin{definition}
	Conținut definiție
\end{definition}

\begin{lemma}
	Conținut lemă.
\end{lemma}

\begin{theorem}
	Conținut teoremă.
	\[
		e^{i * \pi} + 1 = 0
	\]
\end{theorem}

\begin{remark}
	Conținut observație
\end{remark}

Ecuație inline, $2^{10} = 1024 $

\begin{figure}[h]
	\begin{center}
			\includegraphics[width=4cm]{university}
	\end{center}
	\caption{Captură imagine}
\end{figure}